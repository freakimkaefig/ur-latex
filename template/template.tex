\documentclass[11pt,a4paper,oneside,ngerman]{scrartcl} %Deutschsprachiges Dokument mit der passenden Absatzkennzeichnung in der KOMA-Klasse article

\usepackage[T1]{fontenc} %Fontencoding T1 = skalierbare Vektorschrift
\usepackage[ngerman]{babel}
%\usepackage{xunicode}
\usepackage{polyglossia}
\setmainlanguage{german}
\usepackage{xltxtra}
\usepackage{graphicx} %Standard-Paket zum einbinden von Grafiken
\usepackage[bookmarks=true, bookmarksnumbered=false, bookmarksopen=false, colorlinks=true, linkcolor=black, citecolor=black, urlcolor=black]{hyperref}
\usepackage{amsmath}
\usepackage{cancel}
\usepackage{mathcomp}
\usepackage{lmodern,dsfont}
\usepackage{listings}
\usepackage{color}
\usepackage{soul}
\usepackage{setspace}
\usepackage{titlesec}
\usepackage{fancyhdr}
\usepackage{nameref}
\usepackage{microtype} %Verbesserte Trennregeln mithilfe von Mikrotypografie
\usepackage{csquotes} %Paket für kontextsensitive Zitate
\usepackage[output-decimal-marker={,},exponent-product=\cdot]{siunitx} %Paket für Einheiten mitsamt der deutschen Anpassungen
\usepackage[backend=biber, style=apa, language=ngerman, hyperref=true, url=true, backref=true, doi=false, isbn=false, eprint=false, sorting=anyt, maxalphanames=1, labelalpha, natbib=true]{biblatex} %Lade biblatex für die Erstellung der Bibliografie mit biber ohne Sortierung (Einträge werden nach Aufruf sortiert)
\usepackage{tabularx} %Paket für Tabellen mit variabler Breite um diese gleich der Seitenbreite zu setzen
\usepackage[section]{placeins} %Verhindert, dass Abbildungen außerhalbder aktuellen section gesetzt werden
\usepackage[noabbrev]{cleveref} %
\usepackage{epstopdf}
\usepackage{nicefrac}
\usepackage[draft, author={Max Mustermann}]{pdfcomment}
\usepackage{wasysym}
\usepackage{enumitem}

\def\fps@figure{htbp} %Standardpositionierung für Bilder und Tabellen auf htbp anstatt tbp
\def\fps@table{htbp} %Standardpositionierung für Tabellen auf htbp anstatt tbp
\setkeys{Gin}{width=\linewidth,height=\textheight,keepaspectratio} %Bildskalierung auf Seitenbreite
%Bilder zentriert einfügen
\makeatletter
\g@addto@macro\@floatboxreset\centering
\makeatother

%Trennung durch Komma bei mehrfachem \footcite
\usepackage{fnpct}
\AdaptNoteOpt\footcite\multfootcite

\addbibresource{library/literature.bib} %Einbinden der Bibliotheksdatenbank
\DeclareLanguageMapping{ngerman}{ngerman-apa}
%Übersetzungen für Bibliographie
\DefineBibliographyStrings{ngerman}{%
	andothers ={et\addabbrvspace al\adddot}, %et al.
	andmore   ={et\addabbrvspace al\adddot}, %et al.
	nodate    ={n\adddot d\adddot}, %n.d.
}
\addto\extrasngerman{%
	\def\subsectionautorefname{Abschnitt}% Setze den Autorefname von subsection von Unterabschnitt auf Abschnitt
}

% Hurenkinder und Schusterjungen
\clubpenalty10000
\widowpenalty10000
\displaywidowpenalty=10000

% page borders
\usepackage[a4paper,left=3.5cm,right=2.5cm,top=2.5cm,bottom=2.5cm]{geometry}

\newcommand{\getAuthor}{unbekannter Author}
\newcommand{\getTitle}{kein Titel}
\newcommand{\getSemester}{SS 12}
\newcommand{\getErstgutachter}{Kein Dozent}
\newcommand{\getZweitgutachter}{Kein Dozent}
\newcommand{\getDate}{\today}
\newcommand{\getStudID}{}
\newcommand{\getAddress}{}
\newcommand{\getDateHandedIn}{}
\renewcommand{\author}[1]{\renewcommand{\getAuthor}{#1\\}}
\renewcommand{\title}[1]{\renewcommand{\getTitle}{#1}}
\newcommand{\semester}[1]{\renewcommand{\getSemester}{#1}}
\newcommand{\erstgutachter}[1]{\renewcommand{\getErstgutachter}{#1\\}}
\newcommand{\zweitgutachter}[1]{\renewcommand{\getZweitgutachter}{#1\\}}
\newcommand{\studid}[1]{\renewcommand{\getStudID}{#1\\}}
\newcommand{\address}[2]{\renewcommand{\getAddress}{#1, #2\\}}
\newcommand{\dateHandedIn}[1]{\renewcommand{\getDateHandedIn}{#1}}
\newcommand{\writemeta}{
	\hypersetup{pdfinfo={
		Title={\getTitle},
		Author={\getAuthor}
	}}
}

\renewcommand*{\sectionmark}[1]{\markright{#1}}
\renewcommand{\abstract}{\noindent\begin{large}\textbf{Abstract}\end{large}\\\vspace*{10pt}\\}

\newcommand{\N}{\mathds{N}}

\newcommand{\img}[2]{
	\includegraphics[width=#2]{images/#1}
}

\renewcommand{\maketitle}{
\setstretch{1.0}
	
\vspace*{-30pt}
\hfill\includegraphics[width=0.64\textwidth]{images/logo_uni}
\vspace*{20pt}
	
\vspace*{120pt}
	
\begin{center}
	\begin{huge}
		\sffamily\getTitle \\
	\end{huge}
	\vspace*{13pt}
	Masterarbeit im Fach Medieninformatik am \\
	Institut für Medien, Sprache und Kultur (I:IMSK)
\end{center}
	
\vfill

\par
\begingroup
\leftskip=4.9cm
\noindent
Vorgelegt von: \noindent\getAuthor
Adresse: \getAddress
Matrikelnummer: \getStudID
Erstgutachter: Prof. Dr. Erstgutachter\\
Zweitgutachter: Prof. Dr. Zweitgutachter\\
Laufendes Semester: \getSemester\\
Abgegeben am \getDateHandedIn
\par
\endgroup

}

% Code
\definecolor{maroon}{rgb}{0.5,0,0}
\definecolor{darkgreen}{rgb}{0,0.5,0}
\definecolor{gray}{rgb}{0.5,0.5,0.5}


\lstdefinelanguage{XML}
{
	basicstyle=\ttfamily\footnotesize,
	tabsize=2,
	frame=single,
	xleftmargin=15pt,
	framexleftmargin=10pt,
	numbers=left,
	numberstyle=\tiny,
	numbersep=5pt,
	breaklines=true,
	morestring=[s]{"}{"},
	morecomment=[s]{?}{?},
	morecomment=[s]{!--}{--},
	commentstyle=\color{darkgreen},
	moredelim=[s][\color{black}]{>}{<},
	moredelim=[s][\color{red}]{\ }{=},
	stringstyle=\color{blue},
	identifierstyle=\color{maroon}
}
\lstdefinelanguage{JSON}
{
	basicstyle=\ttfamily\footnotesize,
	numbers=left,
	numberstyle=\tiny,
	stepnumber=1,
	numbersep=5pt,
	showstringspaces=false,
	breaklines=true,
	frame=single,
	backgroundcolor=\color{white},
	morecomment=[s]{/*}{*/},
	commentstyle=\color{gray},
	tabsize=2,
	literate=
	 {:}{{{\color{maroon}{:}}}}{1}
	 {,}{{{\color{maroon}{,}}}}{1}
	 {\{}{{{\color{blue}{\{}}}}{1}
	 {\}}{{{\color{blue}{\}}}}}{1}
	 {[}{{{\color{blue}{[}}}}{1}
	 {]}{{{\color{blue}{]}}}}{1}	
}

% Schriftarten
\renewcommand{\familydefault}{\rmdefault}
\setromanfont{Palatino Linotype}
\setmonofont[Scale=1.0,BoldFont={* Medium}]{Source Code Pro}
\setsansfont[BoldFont={* Medium}]{Frutiger Next Pro}
%\setmainfont[BoldFont={* Medium}]{Frutiger Next Pro}

\titleformat*{\section}{\Large\bfseries\sffamily}
\titleformat*{\subsection}{\large\bfseries\sffamily}
\titleformat*{\subsubsection}{\itshape\bfseries\sffamily}

% Keep image inside subsection
\makeatletter
\AtBeginDocument{%
	\expandafter\renewcommand\expandafter\subsection\expandafter{%
		\expandafter\@fb@secFB\subsection
	}%
}
\makeatother

% Keep image inside subsubsection
\makeatletter
\AtBeginDocument{%
	\expandafter\renewcommand\expandafter\subsubsection\expandafter{%
		\expandafter\@fb@secFB\subsubsection
	}%
}
\makeatother